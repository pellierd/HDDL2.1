\def\year{2022}\relax
%File: formatting-instructions-latex-2021.tex
%release 2021.2
\documentclass[letterpaper]{article} % DO NOT CHANGE THIS
\usepackage{aaai22}  % DO NOT CHANGE THIS
\usepackage{times}  % DO NOT CHANGE THIS
\usepackage{helvet} % DO NOT CHANGE THIS
\usepackage{courier}  % DO NOT CHANGE THIS
\usepackage[hyphens]{url}  % DO NOT CHANGE THIS
\usepackage{graphicx} % DO NOT CHANGE THIS
\urlstyle{rm} % DO NOT CHANGE THIS
\def\UrlFont{\rm}  % DO NOT CHANGE THIS
\usepackage{natbib}  % DO NOT CHANGE THIS AND DO NOT ADD ANY OPTIONS TO IT
\usepackage{caption} % DO NOT CHANGE THIS AND DO NOT ADD ANY OPTIONS TO IT
\frenchspacing  % DO NOT CHANGE THIS
\setlength{\pdfpagewidth}{8.5in}  % DO NOT CHANGE THIS
\setlength{\pdfpageheight}{11in}  % DO NOT CHANGE THIS
\usepackage[ruled,linesnumbered,noend]{algorithm2e}
\usepackage{multirow}
\usepackage{subcaption}
\usepackage{amsmath,amssymb,amsthm}
\usepackage{bbm}
\usepackage{subfiles}
\usepackage{listings}
\usepackage{xcolor}


\definecolor{mygreen}{rgb}{0,0.6,0}
\definecolor{mygray}{rgb}{0.5,0.5,0.5}
\definecolor{mymauve}{rgb}{0.58,0,0.82}
\definecolor{myred}{rgb}{0.9,0.2,0.2}


\lstdefinelanguage{pddl}
{
  sensitive=false,    % not case-sensitive
  morecomment=[l]{;}, % line comment
  alsoletter={:,-},   % consider extra characters
  morekeywords={
    define,domain,problem,not,and,or,when,imply,forall,exists,either,
    :domain,:extends,:requirements,:types,:objects,:constants,
    :constraints,:ordered-substasks,:subtasks,:tasks,
    :predicates,:action,:durative-action,:duration,:method,:durative-method,
    :htn,:parameters,:precondition,:condition,:effect,:functions,
    :fluents,:primary-effect,:side-effect,:init,:goal,assign
    :strips,:adl,:equality,:task,:typing,:conditional-effects,:metric,
    :negative-preconditions,:disjunctive-preconditions,
    :existential-preconditions,:universal-preconditions,:ordered-subtasks,:ordering
  },
  keywords=[2]{object,at,start,over,all,end,always,at-most-once,sometime-before,sometime,sometime-after,hold-during,hold-between,hold-after,within,minimize,maximize,total-time} % Objects, temporal elmts
  keywords=[3]{calib_direction,image_direction,instrument,satellite,mode}, % Types
  keywords=[4]{calibrate,turn_approx,turn_precise,take_image,turn_to,activate_instrument,point_to,take_video,method_stereo,do_observation_stereo, do_observation,decrease_overall_quality,method_observe}, % Actions, Methods
  keywords=[5]{observable,calibrated,pointing,supports,power_on,power_avail,on_board,calib_target,have_image,
  image-quality,calib-time,turn-time} % Functions and predicetes
}

\lstset
{
  language={pddl},
  basicstyle=\small\ttfamily, % Global Code Style
  captionpos=b, % Position of the Caption (t for top, b for bottom)
  extendedchars=true, % Allows 256 instead of 128 ASCII characters
  tabsize=2, % number of spaces indented when discovering a tab
  columns=fixed,
  keepspaces=true,
  showstringspaces=false,
  breaklines=true,
  numberstyle=\tiny\ttfamily, % style of the line numbers
  commentstyle=\color{mygrey}, % style of comments
  keywordstyle=[2]\color{mymauve},
  keywordstyle=[3]\color{mygreen},
  keywordstyle=[4]\color{blue},
  keywordstyle=[5]\color{myred},
  stringstyle=\color{blue}, % style of strings
  keywordstyle=\bfseries, % style of keywords
}

\sloppy

% Don't use commands within pdfinfo
%\pdfinfo{
%/Title (An Accurate PDDL Domain Learning Algorithm from Partial and Noisy Observations)
%/Author (Maxence Grand, Damien Pellier, Humbert Fiorino)
%}

\pdfinfo{
/Title (HDDL2.1 Syntax)
/Author (Paper \#)
}

\newtheorem{definition}{Definition}

\newcommand{\at}{\text{\it at}}
\newcommand{\holdafter}{\text{\it hold-after}}
\newcommand{\holdduring}{\text{\it hold-during}}
\newcommand{\before}{\text{\it before}}
\newcommand{\after}{\text{\it after}}
\renewcommand{\between}{\text{\it between}}
\newcommand{\atstart}{\text{\it at start}}
\newcommand{\atend}{\text{\it at end}}
\newcommand{\always}{\text{\it always}}
\newcommand{\overall}{\text{\it overall}}
\newcommand{\sometimes}{\text{\it sometimes}}
\newcommand{\sometimebefore}{\text{\it sometimes-before}}
\newcommand{\sometimeafter}{\text{\it sometimes-after}}
\newcommand{\alwayswithin}{\text{\it always-within}}
\newcommand{\within}{\text{\it within}}
\newcommand{\atmostonce}{\text{\it at-most-once}}

\newcommand{\name}{\text{\it name}}
\newcommand{\pre}{\text{\it precond}}
\newcommand{\param}{\text{\it param}}
\newcommand{\effect}{\text{\it effect}}
\newcommand{\add}{\text{\it effect}^{+}}
\newcommand{\del}{\text{\it effect}^{-}}
\newcommand{\duration}{\text{\it duration}}
\newcommand{\tstart}{\text{\it start}}
\newcommand{\tend}{\text{\it end}}
\newcommand{\tinv}{\text{\it inv}}
\newcommand{\task}{\text{\it task}}
\newcommand{\tn}{\text{\it tn}}
\newcommand{\hbefore}{\text{\it xx }}
\newcommand{\hafter}{\text{\it xx }}

\setcounter{secnumdepth}{2}

\setlength\titlebox{2.5in}
\title{HDDL 2.1 Syntax}
\author{
Damien Pellier, Humbert Fiorino\\
Univ. Grenoble Alpes, LIG\\
38000 Grenoble, France\\
\{Damien.Pellier, Humbert.Fiorino\}@univ-grenoble-alpes.fr}

\title{HDDL 2.1 Syntax - Extended version}
%\author{Paper \#}

\sloppy

\begin{document}
\maketitle

%\section{HDDL 2.1 Syntax}

The proposed syntax, called HDDL2.1, takes its roots in HDDL~\citep{holler20}.

\section{Domain Description}

\newcommand{\specReq}[1]{\ensuremath{\mathtt{^{#1}}}}

The domain definition has been extended to durative actions (line~\ref{l:durative-actions}) and durative methods (line~\ref{l:durative-methods}).
\begin{lstlisting}[escapechar=~]
<domain> ::= (define (domain <name>)
    [<require-def>]
    [<types-def>]~\specReq{:typing}~
    [<predicates-def>]
    [<functions-def>]~\specReq{:fluents}~
    [<constants-def>]
    [<task-defs>]
    <structure-def>*)
\end{lstlisting}

%
% Domain definition
%
% Requirement Statement
% Type Definition
% Domain Constant Definition
% Predicate Definition
%
% @HDDL 1.0

\noindent The definition of the basic domain elements is nearly unchanged.

\begin{lstlisting}[firstnumber=last, escapechar=~]
<require-def> ::= (:requirements  ~\mbox{<require-key>+}~)
<require-key> ::= ~\textit{(see \S\ref{Sec:Requirements})}~
<types-def> ::= (:types ~\mbox{<typed-list(<primitive-type>)>}~)
<predicates-def> ::= (:predicates  ~\mbox{<atomic-formula-skeleton>+}~)
<function-def> ::= (<function-typed-list  ~\mbox{(<atomic-function-skeleton>)}~>)
<constants-def> ::= (:constants <typed-list (constant)>)
<atomic-formula-skeleton> ::= ~\linebreak~(<predicate> <typed-list(variable)>)
<atomic-function-skeleton> ::= ~\linebreak~(<function> <typed-list (variable)>)
<typed-list (x)> ::= x+ - <type> ~\linebreak~[<typed list (x)>]~\label{l:typedlist}~
<type> ::= (either <primitive-type>+)
<type> ::= <primitive-type>
<function-typed-list (x)> ::= x+ - ~\linebreak~<function-type> <function-typed-list(x)>
<function-typed-list (x)> ::=
<function-type> ::=~\specReq{:numeric-fluents}~number
<function-type> ::=~\specReq{:typing} \specReq{+} \specReq{:object-fluents}~<type>
<predicate> ::= <name>
<function> ::= <name>
<constant> ::= <name>
<variable> ::= ?<name>
<primitive-type> ::= <name>
<primitive-type> ::= ~object
\end{lstlisting}

%
% Task definition
%
% @HDDL 1.0
\noindent Abstract tasks are defined similarly to actions as in HDDL 1.0
\begin{lstlisting}[firstnumber=last, escapechar=~]
<tasks-def> ::= (:task <task-def>)
<task-def> ::= <task-symbol> ~\linebreak~:parameters (<typed list (variable)>)~\label{l:compTask}~
<task-symbol> ::= <name>
\end{lstlisting}

\noindent The structure of a temporal HDDL domain is composed of durative and non-durative methods and actions. Durative and non-durative actions are defined as in PDDL and non-durative methods as in HDDL.

\begin{lstlisting}[firstnumber=last, escapechar=~]
<structure-def> ::= <action-def>
<structure-def> ::=~\specReq{:durative-actions}~ ~\mbox{<durative-action-def>}~
<structure-def> ::= <method-def>
<structure-def> ::=~\specReq{:durative-methods}~ ~\mbox{<durative-method-def>}~
\end{lstlisting}



%% Satellite example



% We will continue our description of HTN planning with time using a commercial optical Earth Observing Satellite (EOS) study case.
%, which delimits our requirements for this new language, namely \emph{durative actions} and \emph{durative methods}, \emph{method preconditions}, \emph{numeric preconditions}, and \emph{timed ordering constraints}.
%We present thereafter this real use case of an EOS with the mission of observing areas on the surface of the planet and embedding the capacity of producing and executing autonomous plans.
The example below, grounded in a commercial optical Earth Observing Satellite (EOS) study case, is shaped on the International Planning Competition hierarchical satellite domain~\citep{schattenberg2020}, and is intended to illustrate the temporal features, as well as the orderings and the hierarchical constraints of the proposed language extension.

New generations of commercial EOSs guarantee better performances by embedding a certain degree of autonomy on-board~\citep{pralet2019}. An automated planner is expected to produce new observation mission plans whenever environment changes or new requests are sent from ground control. Because observation tasks depend on the environment ---which is not known in advance--- this is a planning problem and not just a scheduling matter~\citep{rodriguez-moreno2004}.
%
The EOS mission goal is then to fulfil a set of \emph{Acquisition Requests} of images of the Earth surface.
%Requests can be single images, which can be obtained by powering on sensors, and pointing at the sites of interest.
The different modalities of fulfilling those requests require the extension of the satellite domain to a temporal model.

Considering the structure of the requests that must be completed,
the action specifications have a \emph{duration} to reflect the warming up of certain instruments and the time to modify the attitude of the satellite. Figure \ref{pddl_calibrate} shows the calibrate action having a duration specific to the used instrument and defined in the problem initial state.
%
% explication basique
The action has conditions that, if set \texttt{at start} are equivalent to the classical planning PDDL preconditions. In a similar way, effects set to happen \texttt{at end} are applied at the end of the action duration, similarly to classical planning effects.


\begin{figure}[!h]
	\begin{lstlisting}[language=pddl,mathescape]
(:durative-action calibrate
	:parameters (?s - satellite ?i - instrument ?d - calib_direction)
	:duration (= ?duration (calib-time ?i))
	:condition 	(and
			(at start (on_board ?i ?s))
			(at start (calib_target ?i ?d))
			(at start (pointing ?s ?d))
			(at start (power_on ?i)))
	:effect
		(and  (at end (calibrated ?i))) )
	\end{lstlisting}
	\caption{The calibrate action has a duration depending on the instrument calibration time\label{pddl_calibrate}, which are defined in the initial state, e.g.
{\small\lstinline[language={pddl},basicstyle=\ttfamily]|(= (calib-time instrument0) 20)|}.}
\end{figure}



The satellite is agile, that is, it can rotate forwards and backwards in addition to left and right with respect to its track. This capability is used to point towards \textit{sites} that represent the direction of an acquisition request. However this is only possible during a short period of time $[t^{start}, t^{end}]$ in which the location is visible. The observable timespan of each {\tt site} of interest is specified in the initial situation with timed initial literals like in PDDL 2.2~ \citep{hoffmann2005deterministic}, e.g. 
%{\small\lstinline[language={pddl},basicstyle=\ttfamily]|(and (at 500 (observable site))	(at 1000 (not (observable site))))|}.

% as illustrated in Figure~\ref{pddl_initplan_dom}.
 \begin{figure}[!h]
 \begin{lstlisting}[language=pddl]
 (:objects
 	site - image_direction
 )
 (:init
 	(at 500 (observable site))
 	(at 1000 (not (observable site)))
     ...
 ) \end{lstlisting}
 \caption{Timed initial literals for site observability\label{pddl_initplan_dom}}
 \end{figure}

 The \texttt{take-image} preconditions depend then on the site visibility (Figure~\ref{pddl_take-image}). 
 % explication basique
 This action has conditions that should hold at start like in Fig.~\ref{pddl_calibrate}, but also \emph{throughout the action execution}, noted with \texttt{over all}; hence it should be a durative-action, with a duration (here set to a constant value).

\begin{figure}[!h]
\begin{lstlisting}[language=pddl]
(:durative-action take_image
	:parameters (?ti_s - satellite 
	    ?ti_d - image_direction 
	    ?ti_i - instrument ?ti_m - mode)
	:duration (= ?duration 1.00000)
	:precondition (and
		  (over all (observable ?ti_d))
		  (at start (calibrated ?ti_i))
		  (at start (pointing ?ti_s ?ti_d))
		  (over all (on_board ?ti_i ?ti_s))
		  (over all (power_on ?ti_i))
		  (at start (supports ?ti_i ?ti_m))	)
	:effect	(and
		(at end (have_image ?ti_d ?ti_m))
		(at end (increase
		   (image-quality ?ti_s) 2))) )
\end{lstlisting}
\caption{\texttt{take\_image} conditions depend on site observability.\label{pddl_take-image}}
\end{figure}
%

%%%%


%
% Method Definition
%
% @HDDL 1.0
Methods consist of a parameter list (line~\ref{l:mparams}), the abstract task they decompose (line~\ref{l:mabstask}), and the resulting task network (line~\ref{l:msubtasks}). By setting the \verb+:method-preconditions+ requirement, one can use method preconditions (line~\ref{l:mprec}, which can be used for example to filter the hierarchical decomposition).

\begin{lstlisting}[firstnumber=last, escapechar=~]
<method-def> ::= (:method <name>~\label{l:methods}~
    :parameters (<typed list (variable)>)~\label{l:mparams}~
    :task (<task-symbol> <term>*)~\label{l:mabstask}~
    [:precondition <gd>]~\specReq{:method-preconditions}~~\label{l:mprec}~
    <tasknetwork-def>~\label{l:msubtasks}~)
\end{lstlisting}

%
% Durative Method Definition
%
% @HDDL 2.1
Like methods, durative methods have parameters, the abstract task they decompose, and the resulting task network, and, like durative actions, they can have duration constraints and a condition that has to hold to apply the decomposition. Note that, unlike for durative actions, duration constraints are not mandatory. By default the duration of a method is the sum of all the durations of the primitive tasks that compose it. However, it can be interesting to specify temporal constraints on the execution of the decomposition such as, for example, whatever decomposition is chosen, the duration of the primitive tasks must not exceed a given value, or that a subtask must last less than another subtask. These durative constraints can be declared if the requirement {\tt :durative-inequalities} is set. We use here the same requirement as for PDDL.

\begin{lstlisting}[firstnumber=last, escapechar=~]
<durative-method-def> ::=~\label{l:durative-methods}~
    :durative-method <name>
    :parameters (<typed list (variable)>)
    :task (<task-symbol> <term>*)
    [:duration <method-duration-constraint>]~\specReq{:duration-inequalities}~
    [:condition <da-gd>]~\specReq{:method-preconditions}~~\label{l:mcond}~
    <tasknetwork-def>~\label{l:msubtasks2}~)
\end{lstlisting}


%%% SAT example

Pointing to sites is performed by the \texttt{turn\_to} task. \texttt{turn\_to} can be decomposed in two ways, one of which is less precise (the overall quality of the images will be reduced) but quicker to execute. These two \emph{durative-methods} differ mainly in the duration constraints, reflecting the different pointing times, as shown in Figure~\ref{pddl_durative_m}. This choice, only possible on-board, corrects over-confident plans regarding transition times that otherwise would have caused the acquisition to fail. The planning problem can be designed to optimise image quality, but also accept lower quality images when time is short. On the modelling side, the distinction between the two methods is done
on the basis of the duration interval.
%
\begin{figure}[!h]
\begin{lstlisting}[language=pddl]
(:durative-method turn_precise
	:parameters (?tp_s - satellite 
	    ?tp_d_prev - image_direction 
	    ?tp_d - image_direction)
	:task (turn_to ?tp_s ?tp_d ?tp_d_prev)
	:duration (>= ?duration (* (/ (turn-time 
	    (t_d_new t_d_prev)) 4) 5) ) 
	:ordered-subtasks (and
		(point_to ?tp_s ?tp_d ?tp_d_prev)
		(take_image ?tp_s ?tp_d ?tp_i ?tp_m) )
	:constraints (and
		(not (= ?tp_d ?tp_d_prev))) )

(:durative-method turn_approx
	:parameters (?ta_s - satellite 
	    ?ta_d_prev - image_direction 
	    ?ta_d - image_direction)
	:task (turn_to ?ta_s ?ta_d ?ta_d_prev)
	:duration (and
		  (<= ?duration (* (/ (turn-time 
		      (t_d_new t_d_prev)) 4) 5)) 
		  (>= ?duration (/ (turn-time 
		      (t_d_new t_d_prev)) 2) ) 
	:ordered-subtasks (and
		(point_to ?ta_s ?ta_d ?ta_d_prev)
		(take_image ?ta_s ?ta_d ?ta_i ?ta_m) 
		(decrease_overall_quality ?ti_s) )
	:constraints (and
		(not (= ?ta_d ?ta_d_prev))) )
\end{lstlisting}
\caption{\texttt{turn\_precise} and \texttt{turn\_approx} methods use duration inequalities to reflect the duration time, which is bounded by the duration of the \emph{durative-action} \texttt{point\_to}. The duration of \texttt{point\_to} is bounded by the relative position of the two sites to observe.\label{pddl_durative_m}}
\end{figure}


%%%%%%
%
% Task Definition
%
% @HDDL 1.0
The definition of task networks is used in method definitions as well as in the problem definition to define the initial task network. It contains the definition of sub-tasks (line~\ref{l:tnsubtasks}), ordering constraints (line~\ref{l:tnordering}), and logical constraints (line~\ref{l:tnconstraints}) on the sub-tasks of the method. We keep here the same definition as in HDDL.

When the key \verb+:ordered-subtasks+ is used, the network is considered totally ordered. In the other cases, ordering relations have to be defined explicitly. This is done by including ids into the task definition that can then be referenced in the ordering definition.

\begin{lstlisting}[firstnumber=last, escapechar=~]
<tasknetwork-def> ::=
    [~\textbf{:}~[~\textbf{ordered-}~][~\textbf{sub}~]~\textbf{tasks}~
        <subtask-defs>]~\label{l:tnsubtasks}~
    [~\textbf{:order}~[~\textbf{ing}~] <ordering-defs>]~\label{l:tnordering}~
    [:constraints <constraint-defs>]~\label{l:tnconstraints}~
\end{lstlisting}

%
% Subtasks
%
% @HDDL 1.0
\noindent The subtask definition can contain one or more subtasks.

\begin{lstlisting}[firstnumber=last, escapechar=~]
<subtask-defs> ::= () | <subtask-def> ~\linebreak~| (and <subtask-def>+)
<subtask-def> ::= (<task-symbol> <term>*) ~\linebreak~| (<subtask> (<task-symbol> <term>*))
<subtask> ::= <name>
\end{lstlisting}

%
% Ordering
%
% @HDDL 2.1
The ordering constraints are defined via the task ids, and have to induce a partial order. To deal with time we add time specifiers on task ids to specify if the ordering constraints are about the start time or the end time of the task. The time specifiers are like in PDDL: ~{\tt start} to define the start time of a task, and {\tt end}  to define the end time of a task. Note that we extend also the list of operators available to express ordering constraints between tasks. It has been extended to define any constraints on the start or end of a sub-task of a method, and allows to express all Allen's intervals \citep{allen81} between these subtasks.

\begin{lstlisting}[firstnumber=last, escapechar=~]
<ordering-defs> ::= () | <ordering-def> ~\linebreak~| (and <ordering-def>+)
<ordering-def> ::= ~\linebreak~(<d-task> <task-id> <task-id>)
<ordering-def> ::=~\specReq{:durative-action} \linebreak~(not (<ordering-def>))
<d-task> ::= <
<d-task> ::=~\specReq{:durative-action} ~>=
<d-task> ::=~\specReq{:durative-action} ~<=
<d-task> ::=~\specReq{:durative-action} ~>
<d-task> ::=~\specReq{:durative-action} ~=
<task-id> ::= <name>
<task-id> ::= ~\specReq{:durative-action} <time-task-id>~
<time-task-id> ::= (<time-specifier> ~\mbox{<task-id>)}~
<time-specifier> ::= ~start~
<time-specifier> ::= ~end
\end{lstlisting}

%
% Constraints extensions
%
% @HDDL 2.1
HDDL~1.0 only accepts equality constraints or inequality on method parameters in the task network. To increase the expressiveness of the language, we have chosen to add two kinds of constraints: the {\em classical decomposition constraints} ($\before$, $\after$, $\between$) used in HTN planning introduced first by \citep{erol94} to represent constraints between tasks for non temporal HTN problem and the {\em temporal decomposition constraints} based on the plan trajectory constraint from PDDL~3.0. The semantic of theses constraints were given in section~\ref{Sec:Decomposition-Constraints-Semantics}. To allow the specification of such the constraints added, it is necessary to use \verb+:method-constraints+.

\begin{lstlisting}[firstnumber=last, escapechar=~]
<constraint-def> ::= () | <constraint-def> | (and <constraint-def>+)
<constraint-def> ::= (not (= <term> <term>))~\linebreak~| (= <term> <term>)
<constraint-def> ::=~\specReq{:method-constraints}~  ~\linebreak~(~before <task-id> <gd>)~
<constraint-def> ::=~\specReq{:method-constraints}~  ~\linebreak~(~after <task-id> <gd>)~
<constraint-def> ::=~\specReq{:method-constraints}~  ~\linebreak~(~between <task-id> <task-id> <gd>)~
<constraint-def> ::=~\specReq{:durative-action + :method-constraints}~ ~\linebreak~<timed-constraint-def>
<timed-constraint-def> ::= ~\linebreak~(~at <timed-task-id> <gd>)~
<timed-constraint-def> ::= ~\linebreak~(~hold-during <timed-task-id> <gd>)~
<timed-constraint-def> ::= ~\linebreak~(~within <timed-task-id> <gd>)~
<timed-constraint-def> ::= ~\linebreak~(~hold-after <timed-task-id> <gd>)~
<timed-constraint-def> ::= ~\linebreak~(~at start <gd>)~
<timed-constraint-def> ::= ~\linebreak~(~at end <gd>)~
<timed-constraint-def> ::= ~\linebreak~(~always <gd>)~
<timed-constraint-def> ::= ~\linebreak~(~at-most-once <gd>)~
<timed-constraint-def> ::= ~\linebreak~(~sometime <gd>)~
<timed-constraint-def> ::= ~\linebreak~(~sometime-before <gd> <gd>)~
<timed-constraint-def> ::= ~\linebreak~(~sometime-after <gd> <gd>)~
<timed-constraint-def> ::= ~\linebreak~(~always-within number <gd> <gd>)~
\end{lstlisting}



Figure \ref{pddl_durative_m_con} depicting the \textit{observe} method illustrates some temporal constraints semantics introduced here.
% section \ref{ssec:temporal-constraints-semantics}.
In this case, \texttt{pointing} at an observation site can be achieved by different subtasks. By requiring that this precondition of \mbox{\small\lstinline[language={pddl},basicstyle=\ttfamily]|durative-action take_image|} is performed  at most once, avoids useless satellite movements.
Note that, besides the explicit subtask ordering, an implicit ordering between tasks 0 and 1 could have been forced by adding the constraint  \mbox{\small\lstinline[language={pddl},basicstyle=\ttfamily]|(within task1 (power_on ?i))|}, requiring that the instrument must be powered on \emph{before} turning towards the observation site.
%
\begin{figure}[h!]
	\begin{lstlisting}[language=pddl, basicstyle=\fontsize{8.5}{10}\selectfont\ttfamily, escapechar=~]
	(:durative-method method_observe
		:parameters (?d_prev - image_direction
		   ?d - image_direction
		   ?s - satellite
		   ?i - instrument   ?m - mode)
		:task (do_observation ?d ?m)
		:duration
		   (< (duration task1) (calib-time ?i))~\label{l:mdur}~
		:subtasks (and
		   (task0 (activate_instrument ?s ?i))
		   (task1 (turn_to ?s ?d ?d_prev))
		   (task2 (take_image ?s ?d ?i ?m)))
		:ordering (and
		   (< task0 task2)
		   (< task1 task2))
		:constraints (and
		   (not (= ?d ?d_prev))
		   (at-most-once (pointing ?s ?d))) )~\label{l:mhb}~
	\end{lstlisting}
	\caption{\texttt{method\_observe} uses a duration inequality and a constraint over literals to constrain an ordering. Similarly, a decomposition can be constrained at method-level.\label{pddl_durative_m_con}}
\end{figure}


%
% Action Definition
%
% HDDL 1.0
The orginal action or durative action definitions defined in PDDL stay mainly unchanged. The only difference is that for simplicity, it is not possible to specify preferences. The non-durative actions are defined as follows:

\begin{lstlisting}[firstnumber=last, escapechar=~]
<action-def> ::= (:action <task-def>~\label{l:action}~
    [:precondition <gd>]
    [:effect <effect>])
\end{lstlisting}

%
%  Durative Action Definition
%
% PDDL 3.0 add to HDDL 2.1
\noindent The durative actions are defined as follows:

\begin{lstlisting}[firstnumber=last, escapechar=~]
<durative-action-def> ::= ~\linebreak~(:durative-action <task-def>~\label{l:durative-actions}~
      :duration <duration-constraint>
      [:condition <da-gd)>]
      [:effect <da-effect>])
\end{lstlisting}


%
%  Durative Action Constraints Definition
%
% PDDL 3.0 add to HDDL 2.1
As in PDDL, duration constraints with the \verb|:duration-inequalities| requirement allow to express duration inequalities. The definition of the duration constraints for durative actions does not change.

\begin{lstlisting}[firstnumber=last, escapechar=~]
<action-duration-constraint> ::=~\specReq{:duration-inequalities}~(and ~\linebreak~ <simple-action-duration-constraint>+)
<action-duration-constraint> ::= ()
<action-duration-constraint> ::= ~\mbox{<simple-action-duration-constraint>}~
<action-simple-duration-constraint>::= ~\linebreak~(<d-op> ?duration <d-value>)
<action-simple-duration-constraint>::= ~\linebreak~ (~{at}~ <time-specifier> ~\mbox{\hspace{2em}<simple-duration-constraint>}~)
<d-op> :: <=
<d-op> :: >=
<d-op> :: =
<d-value> ::= <number>
<d-value> ::=~\specReq{:numeric-fluents}~<f-exp>
<f-exp> ::=~\specReq{:numeric-fluents}~<number>
<f-exp> ::=~\specReq{:numeric-fluents}~(<binary-op> <f-exp> <f-exp>)
<f-exp> ::=~\specReq{:numeric-fluents}~(<multi-op> <f-exp> <f-exp>+)
<f-exp> ::=~\specReq{:numeric-fluents}~(- <f-exp>)
<f-exp> ::=~\specReq{:numeric-fluents}~<f-head>
<f-head> ::= (<function-symbol> <term>*)
<f-head> ::= <function-symbol>
<binary-op> ::= <multi-op>
<binary-op> ::= -
<binary-op> ::= /
<multi-op> ::= *
<multi-op> ::= +
<binary-comp> ::= >
<binary-comp> ::= <
<binary-comp> ::= =
<binary-comp> ::= >=
<binary-comp> ::= <=
\end{lstlisting}

%
%  Durative Method Constraints Definition
%
% @HDDL 2.1
The duration of a method is by default the sum of the durations of the sub-tasks that compose it. Optional constraints might be specified to restrict the allowed value for the
duration of the method or the sub-tasks that compose it. Therefore, it is necessary to be able to refer explicitly to the duration of a subtask and not only to the duration of the method with the variable {\tt ?duration}.

\begin{lstlisting}[firstnumber=last, escapechar=~]
<method-duration-constraint> ::=~\specReq{:duration-inequalities}~ (and~\linebreak~ <method-simple-duration-constraint>+)
<method-duration-constraint> ::= ()
<method-duration-constraint> ::= ~\mbox{<simple-method-duration-constraint>}~
<simple-method-duration-constraint>::= ~\linebreak~(<binary-comp> <duration> <td-value>)
<simple-method-duration-constraint>::= ~\linebreak~(~{at}~ <time-specifier> ~\mbox{\hspace{2em}<simple-method-duration-constraint>}~)
<duration> :: = ?duration
<duration> :: = <task-duration>
<td-value> ::= <d-value>
<td-value> ::= <task-duration>
<task-duration> :: = (duration <task-id>)
\end{lstlisting}

%
% Goal Description
%
% @HDDL 1.0
\noindent Compared to HDDL 1.0 we allow to define numerical and temporal preconditions. The syntax used from numeric preconditions is the same as that of PDDL 3.0. For the sake of simplicity, we chose to not include the preferences.
\begin{lstlisting}[firstnumber=last, escapechar=~]
<gd> ::= ()
<gd> ::= <literal (term)>
<gd> ::= (and <gd>*)
<gd> ::=~\specReq{:disjunctive-preconditions}~ (or <gd>*)
<gd> ::=~\specReq{:negative-preconditions}~ (not <gd>)
<gd> ::=~\specReq{:disjunctive-preconditions \ :negative-preconditions}~ ~\mbox{(imply <gd> <gd>)}~
<gd> ::=~\specReq{:existential-preconditions}~ ~\linebreak~(exists (<typed list (variable)>*) <gd>)
<gd> ::=~\specReq{:universal-preconditions}~ ~\linebreak~(forall (<typed list (variable)>*) <gd>)
<gd> ::= (= <term> <term>)
<gd> ::=~\specReq{:numeric-fluents} <f-comp>
<f-comp> ::= (<binary-comp> <f-exp> <f-exp>)
\end{lstlisting}

\begin{lstlisting}[firstnumber=last, escapechar=~]
<literal (t)> ::= <atomic formula(t)>
<literal (t)> ::= (not <atomic formula(t)>)
<atomic formula(t)> ::= (<predicate> t*)
\end{lstlisting}

\begin{lstlisting}[firstnumber=last, escapechar=~]
<term> ::= <name>
<term> ::= <variable>
\end{lstlisting}

%
% Goal Description
%
% PDDL 3.0 add to HDDL 2.1

\noindent The same approach is used for temporal precondition definition. The syntax used is the same as in PDDL 3.0.

\begin{lstlisting}[firstnumber=last, escapechar=~]
<da-gd> ::= <timed-gd>
<da-gd> ::= (and <da-gd>*)
<da-gd> ::=~\specReq{:universal-preconditions}~ (forall ~\linebreak~ <typed-list (variable)>) <da-gd>)
<timed-gd> ::= (~at <time-specifier> <gd>)~
<timed-gd> ::= (~over <interval> <gd>)~
<interval> ::= ~all
\end{lstlisting}

%
% Effects
%
% @HDDL 1.0
\noindent Symmetrically, we add to the effect definition the temporal and numeric aspect based on the same syntax as in PDDL 3.0.

\begin{lstlisting}[firstnumber=last, escapechar=~]
<effect> ::= ()
<effect> ::= (and <c-effect>*)
<effect> ::= <c-effect>
<c-effect> ::=~\specReq{:conditional-effects}~ ~\linebreak~(forall (<variable>*) <effect>)
<c-effect> ::=~\specReq{:conditional-effects}~ ~\linebreak~(when <gd> <cond-effect>)
<c-effect> ::= <p-effect>
<p-effect> ::= (not <atomic formula(term)>)
<p-effect> ::= <atomic formula(term)>
<cond-effect> ::= (and <p-effect>*)
<cond-effect> ::= <p-effect>
\end{lstlisting}

%
%  Temporal Method Definition
%
% From PDDL 3.0 add to HDDL 2.1

\noindent The temporal and numeric definitions of effects are as follows:

\begin{lstlisting}[firstnumber=last, escapechar=~]
<da-effect> ::= (and <da-effect>*)
<da-effect> ::= <timed-effect>
<da-effect> ::=~\specReq{:conditional-effects}~ (forall ~\linebreak~(<typed list (variable)>) <da-effect>)
<da-effect> ::=~\specReq{:conditional-effects}~ ~\linebreak~(when <da-gd> <timed-effect>)
<timed-effect> ::= ~\linebreak~(~at <time-specifier> <cond-effect>)~
<timed-effect> ::=~\specReq{:numeric-fluents}~ ~\linebreak~(~at <time-specifier> <f-assign-da>)~
<f-assign-da> ::= (<assign-op> <f-head>~\linebreak ~<f-exp-da>)
<f-exp-da> ::= (<binary-op> <f-exp-da>~\linebreak ~ <f-exp-da>)
<f-exp-da> ::= (<multi-op> <f-exp-da>~\linebreak ~ <f-exp-da>+)
<f-exp-da> ::= (- <f-exp-da>)
<f-exp-da> ::=~\specReq{:duration-inequalities}~ ?duration
<f-exp-da> ::= <f-exp>
<assign-op-t> ::= increase
<assign-op-t> ::= decrease
\end{lstlisting}

% continuous-effects rules removed
%<timed-effect> ::=~\specReq{:continuous-effects + :numeric-fluents}~ ~\linebreak~(<assign-op-t> <f-head> <f-exp-t>)
%<f-exp-t> ::= (* <f-exp> #t)
%<f-exp-t> ::= (* #t <f-exp>)
%<f-exp-t> ::= #t

\section{Problem Description}

%
% Problem Definition
%
% @HDDL 1.0 + initial timed literal + initial function values + metric spec
The problem definition includes as additional element the initial task network (line~\ref{l:tnihtn}) as in HDDL 1.0. But now, it is possible to define in the initial state of the problem initial function values and initial time literals. It is also possible to define metric specs on solution plans. We have chosen here not to allow the definition of preferences. But preferences could be added in a next extension of the language. Likewise we have also chosen not to allow the definition of constraints on plans although it is possible in PDDL. We believe that it is preferable to add constraints on plans by adding logical constraints associated with task networks. This allows for a more unified language.

\begin{lstlisting}[firstnumber=last, escapechar=~]
<problem> ::= (define (problem <name>)
    (:domain <name>)
    [<require-def>]
    [<object-declaration>]
    [<htn>]~\label{l:tnihtn}~
    <init>
    [<goal>])~\label{l:goal}~
    [<metric-spec>]~\specReq{:numeric-fluents}
\end{lstlisting}

\begin{lstlisting}[firstnumber=last, escapechar=~]
<object-declaration> ::= ~\linebreak~(:objects <typed list (name)>)
<init> ::= (:init <init-el>*)
<init-el> ::= <literal (name)>
<init-el> ::=~\specReq{:timed-initial-literals}~(~at ~<number>~ <literal (name)>)~
<init-el> ::=~\specReq{:numeric-fluents}~ (= ~<basic-function-term>~ <number>)
<basic-function-term> ::= <function-symbol>
<basic-function-term> ::= (<function-symbol> <name>*)
<goal> ::= (:goal <gd>)
\end{lstlisting}

The initial task network contains the definition of the problem class.

\begin{lstlisting}[firstnumber=last, escapechar=~]
<htn> ::= (:htn
    [:parameters (<typed list (variable)>)] ~\label{l:tniparams}~
    <tasknetwork-def>)~\label{l:tnitasks}~
\end{lstlisting} %| :tihtn~\specReq{:tihtn}~

%
% Metric spec
%
% @HDDL 2.1

The optional metric spec can be defined using the same syntax as in PDDL 3.0.

\begin{lstlisting}[firstnumber=last, escapechar=~]
<metric-spec> ::=~\specReq{:numeric-fluents}~
    (:metric <optimization> <metric-f-exp>)
<optimization> ::= ~minimize~
<optimization> ::= ~maximize~
<metric-f-exp> ::= (<binary-op>
    <metric-f-exp> <metric-f-exp>)
<metric-f-exp> ::= (<multi-op>
    <metric-f-exp> <metric-f-exp>+)
<metric-f-exp> ::= (- <metric-f-exp>)
<metric-f-exp> ::= <number>
<metric-f-exp> ::= ( <function-symbol>
    <name>* )
<metric-f-exp> ::= <function-symbol>
<metric-f-exp> ::= ~total-time
\end{lstlisting}

\section{Temporal and Numeric HDDL Requirements}
\label{Sec:Requirements}
The overall definition includes all the following requirement flags as in HDDL 1.0:
\begin{itemize}
 \item \verb+:hierarchy+ requires the applied system to support HTN planning, so this can be seen as the basic requirement for the language defined here.
 \item \verb+:method-preconditions+ requires the applied system to support method preconditions or condition when define in durative methods.
\end{itemize}
But now, the following requirement flags are also compatible with HDDL 2.1:
\begin{description}
 \item \verb+:durative-actions+ requires the applied system to support durative actions and temporal ordering constraints in method definitions.
 \item \verb+:duration-inequalities+ requires the applied system to support duration inequalities in durative actions declarations. Implies \verb+:durative-actions+.
 \item \verb+:timed-initial-literals+ requires the applied system to support initial state with literals that become true at a specified time point. Implies \verb+:durative-actions+.
 \item \verb+:numeric-fluent+ requires the applied system to support numeric fluents in preconditions and effects of actions and methods.
 %\item \verb+:continuous-effects+ requires the applied system to support continuous action effects.
 \end{description}
 We add the following requirement flag:
\begin{description}
\item \verb+:method-constraints+ requires to specify decomposition constraints in methods.
 \end{description}


\end{document}
